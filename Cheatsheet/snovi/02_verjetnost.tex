\section{Verjetnost}

\subsection{Elementarna verjetnost}
$\Omega \equiv\text{elementarna verjetnost}$\\
Izidi morajo biti enako vrjetni!\\
$A \subseteq \Omega \equiv\text{dogodek}$\\ 
$P(A)\in [0, 1]$\\
$P(\emptyset)=0$\\
$P(\Omega)=1$\\
$P(A)=\frac{|A|}{|\Omega|}$\\
Uporabne zveze:\\
$P(A\cup B)=P(A)+P(B)-P(A\cap B)$\\
$P(A\cap B)=P(A)+P(B)-P(A\cup B)$\\
$P(A^c)=1-P(A)$\\


\subsection{Geometrijska verjetnost}
$P(A)=\frac{\text{mera }A}{\text{mera }B}$\\
$\text{Mera: dolzina, ploscina, volumen}$\\

\subsection{Neovdisni dogodki}
Ce sta dogodka A in B neodvisna:\\
$P(A\cap B)=P(A)\cdot P(B)$\\

\subsection{Nacelo vkljucitev in izkljucitev}
$P(A_1\cup ... \cup A_2)=P(A_1)+...+P(A_n)\\-P(A_1A_2)-P(A_1A_3)-...-P(A_{n-1}A_n)\\+P(A_1A_2A_3)+P(A_1A_2A_4)+...+P(A_{n-2}A_{n-1}A_n)\\-...\\+(-1)^{n+1}P(A_1...A_n)$\\
\\
$P(A\cup B \cup C)=\\P(A)+P(B)+P(C)\\-P(A\cap B)-P(A\cap C)-P(B\cap C)\\+P(A\cap B\cap C)$\\
