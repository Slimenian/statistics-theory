\section{CLI}

\subsection{Vsota normalnih spremenljivk}
Naj bodo:\\
$X_1\sim N(\mu_1,\sigma_1),\; X_2\sim N(\mu_2,\sigma_2),\; \ldots,\; X_n\sim N(\mu_n,\sigma_n)$\\
\textbf{neodvisne}. Potem velja:\\
$$X_1+X_2\ldots + X_n\sim N\left( \sum_{i=1}^n \mu_i, \sqrt{\sum_{i=1}^n \sigma_i^2}\right).$$

\subsection{Centralni limitni izrek}
Naj bodo $X_1$, ... $X_n$ \underline{neovidsne} in \underline{enako porazdeljene} slucajne spremenljivke s pricakovoano vrednostjo $\mu$ in standardnim odklonom $\sigma$.
Potem za dovolj velik $n$ velja, da je porazdelitev vsote $S=X_1+X_2+\dots X_n$ priblizno normalna:\\
$$S\sim N(n\cdot \mu, \sigma \cdot \sqrt{n})$$

Pri aproksimaciji \underline{diskretne porazdelitve vsote} uporabljamo popravek za zvezonst $0.5$\\
$P(a\leq S\leq b)\approx P(a-0.5\leq Y\leq b+0.5)=F(b+0.5)-F(a-0.5)$\\

\subsection{Normalno porazdeljen vzorec}
\textbf{Enostavni slucajni vzorec} je slucajni vekotor $(X_1,...,X_n)$ za katerega velja:
\begin{itemize}[leftmargin=*]
    \item vsi $X_i$ imajo enako porazdelitev
    \item $X_i$ so med sabo neodvisni
\end{itemize}
\textbf{Vzorcno povprecje normalno porazdeljenega vzorca}
Naj bo enostavni $(X_1,...,X_n)$ slucajni vzorec, kjer $X_i\sim N(\mu, \sigma)$. Potem je:\\
$\overline{X}=\frac{1}{n}\sum\limits^n_{i=1} X_i$ tudi normalna\\
$$\overline{X}\sim N(\mu, \frac{\sigma}{\sqrt{n}})$$


\subsection{Centralni limitni izrek za vzorncno povprecje}
Naj bo enostavni $(X_1,...,X_n)$ slucajni vzorec in:\\
$E(X_i)=\mu$ ter  $D(X_i)=\sigma^2 < \infty$\\
potem za dovolj veliki vzorec $n \ge 30$:\\
$$\overline{X}\sim N (\mu, \frac{\sigma}{\sqrt{n}})$$
