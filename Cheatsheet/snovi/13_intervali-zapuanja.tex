\section{Intervali zaupanja}
Naj bo $(X_1, X_2, ..., X_n)$ prost slucajni vzorec.\\
Za $(x_1, x_2, ..., x_n)$ oznacimo dobljeni vzorec konkretnih vrednosti.

\begin{center}
\scalebox{0.8}{
\begin{tabular}{|c|c|c|}
\hline
Parameter & Cenilka $f(X_1,..,X_n)$ & Ocena $f(x_1,..x_n)$ \\
\hline
$\mu$ & $\overline{X}=\frac{1}{n}\sum\limits^n_{i=1}X_i$ & $\overline{x}=\frac{1}{n}\sum\limits^n_{i=1}x_i$\\
\hline
$\sigma$ & $S=\sqrt{\frac{1}{n-1}\sum\limits^n_{i=1}(X_i-\overline{X})^2}$ & $S=\sqrt{\frac{1}{n-1}\sum\limits^n_{i=1}(x_i-\overline{x})^2}$\\
\hline
p & $\overline{X}=\frac{1}{n}\sum\limits^n_{i=1}X_i$ & $\overline{X}=\frac{1}{n}\sum\limits^n_{i=1}X_i$ \\
\hline
\end{tabular}}
\end{center}

\subsection{Interval zaupanja za $\mu$ ($\sigma$ je znan)}
Za konstrukcijo intervala zaupanja uporabljamo dejstvo\\
$$\frac{\overline{X}-\mu}{\frac{\sigma}{\sqrt{n}}}\sim N(0,1)$$\\
Slucajna spremenljivka X je normalno porazdeljena ali imamo dovolj veliki vzorec (za uporabo CLI).\\
Z verjetnostjo $1-\alpha$ se $\mu$ nahaja v intervalu $$[\overline{X}-\epsilon, \overline{X}+\epsilon]$$\\
Dobimo $$\epsilon=c\frac{\sigma}{\sqrt{n}}$$, kjer je $\underline{c=F^{-1}(1-\frac{\alpha}{2})}$ (kvantil standardne normalne porazdelitve)\\
Interval zaupanja za $\mu$\\
$$I_\mu = \left[\overline{X}-c\frac{\sigma}{\sqrt{n}},\overline{X}+c\frac{\sigma}{\sqrt{n}}\right]$$    


\subsection{Interval zaupanja za $\mu$ ($\sigma$ ni znan)}
Za konstrukcijo intervala zaupanja uporabimo dejstvo:\\
$$\frac{\overline{X}-\mu}{\frac{S}{\sqrt{n}}}\sim t_{n-1}$$
Slucajna spremenljivka X je normalno porazdeljena ali imamo dovolj veliki vzorec.\\
Interval zaupanja za $\mu$ stopnjo zaupanja $1-\alpha$ je enak\\
$$I_\mu = \left[ \overline{X}- c\frac{S}{\sqrt{n}}, \overline{X}+ c\frac{S}{\sqrt{n}}\right]$$\\
kjer je $c=t_{n-1;1-\frac{\alpha}{2}}$ kvantil Studentovo porazdelitve z n-1 prostosnimi stopnjami.\\


\subsection{Interval zaupanja za delez p}
Za konstrukcijo intervala zaupanja uporabljamo dejstvo:\\
$$\hat{p}\sim N(p, \sqrt{\frac{p(1-p)}{n}})$$
p je delez populacije z doloceno lastnostjo.\\
Naj bo $(X_1,...,X_n)$ enostavni slucajni vzorec, kjer je $X_i$ indikatorska slucajna spremenljivka.
\begin{equation*}
X_i \sim
\begin{pmatrix}
   0 & 1\\
   1-p & p 
\end{pmatrix}
\end{equation*}
Neznani delez p ocenjujemo z vzorcnim delezom.\\
$$ \hat{p} \sim \overline{X}=\frac{1}{n}\sum\limits^n_{i=1} X_i$$
$$I_p = \left[\hat{p}-c \sqrt{\frac{\hat{p}(1-\hat{p})}{n}}, \hat{p} + c \sqrt{\frac{\hat{p}(1-\hat{p})}{n}}\right]$$
$c=F^{-1}(1-\frac{\alpha}{2})$

